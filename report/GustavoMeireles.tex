\documentclass{article}
\usepackage{fullpage}
\usepackage{hyperref}
\usepackage{listings}
\usepackage{color}

% javascript colors
\definecolor{lightgray}{rgb}{.9,.9,.9}
\definecolor{darkgray}{rgb}{.4,.4,.4}
\definecolor{purple}{rgb}{0.65, 0.12, 0.82}

\lstdefinelanguage{JavaScript}{
  keywords={typeof, new, true, false, catch, function, return, null, catch, switch, var, if, in, while, do, else, case, break},
  keywordstyle=\color{blue}\bfseries,
  ndkeywords={class, export, boolean, throw, implements, import, this},
  ndkeywordstyle=\color{darkgray}\bfseries,
  identifierstyle=\color{black},
  sensitive=false,
  comment=[l]{//},
  morecomment=[s]{/*}{*/},
  commentstyle=\color{purple}\ttfamily,
  stringstyle=\color{red}\ttfamily,
  morestring=[b]',
  morestring=[b]"
}

\lstset{
   language=JavaScript,
   backgroundcolor=\color{lightgray},
   extendedchars=true,
   basicstyle=\footnotesize\ttfamily,
   showstringspaces=false,
   showspaces=false,
   numbers=left,
   numberstyle=\footnotesize,
   numbersep=9pt,
   tabsize=2,
   breaklines=true,
   showtabs=false,
   captionpos=b
}

\title{Relatório do Meu Projeto: Portfolio Pessoal}
\author{Aluno: Gustavo Meireles}
\date{Data de Entrega: \today}

\begin{document}
\maketitle

\tableofcontents
\newpage

\section{Introdução}
Este relatório visa descrever o projeto do meu website pessoal, no qual apresento o meu portfólio e informações sobre mim. A missão principal do website é mostrar as minhas habilidades como desenvolvedor e compartilhar os projetos em que trabalhei. Nesta parte, irei detalhar as tecnologias, frameworks e funcionalidades que utilizei para criar o website.

\section{Missão e Objetivos}
O meu website tem como missão principal apresentar o meu portfólio e destacar o meu trabalho como desenvolvedor. Os objetivos específicos incluem:

\begin{itemize}
    \item Apresentar informações pessoais e de contato.
    \item Destacar projetos de web development e .NET.
    \item Apresentar os meus projetos pessoais de scripts Bash.
    \item Fornecer uma experiência de usuário interativa e responsiva.
    \item Conectar-me a potenciais empregadores ou colaboradores.
\end{itemize}

\section{Detalhes Técnicos}
Nesta secção, irei explicar as tecnologias e componentes técnicos que utilizei para desenvolver o website.

\subsection{Tecnologias Front-End}
Para o desenvolvimento do front-end do meu website, utilizei as seguintes tecnologias:

\begin{itemize}
    \item HTML para a estrutura do conteúdo.
    \item CSS para a estilização e layout.
    \item JavaScript para tornar o site interativo.
    \item Framework jQuery para facilitar a interação com elementos da página.
    \item Font Awesome para ícones.
\end{itemize}

\subsection{Estrutura do Site}
O meu website está organizado em várias seções de conteúdo, incluindo:

\begin{itemize}
    \item Cabeçalho: Apresenta o meu nome e pontos-chave.
    \item Sobre Mim: Contém uma breve introdução sobre mim.
    \item Projetos: Destaca os meus projetos de web development e .NET.
    \item Projetos Pessoais: Apresenta os meus scripts Bash pessoais.
    \item Rodapé: Inclui informações de contato e links para redes sociais.
\end{itemize}

\subsection{Funcionalidades Interativas}
Para tornar o meu website mais interativo, implementei as seguintes funcionalidades:

\begin{itemize}
    \item Efeitos de Hover: Adicionei efeitos de hover a títulos, parágrafos e bullet points para fornecer feedback visual quando os visitantes passam o cursor sobre eles.
    \item Carrossel de Imagens: Implementei um carrossel de imagens para apresentar detalhes de projetos. Os visitantes podem clicar nas imagens para alternar entre diferentes visualizações.
    \item Atualização Automática de Bullet Points: Os bullet points na parte superior da página são atualizados automaticamente a cada poucos segundos para exibir mensagens diferentes.
\end{itemize}

\subsection{Funcionalidades Interativas}
Para tornar o meu website mais interativo, implementei as seguintes funcionalidades:

\subsubsection{Efeito de Destaque em Elementos}
Implementei um efeito de destaque em elementos da página que é ativado quando os visitantes passam o cursor sobre eles. Para isso, utilizei o jQuery, que é uma biblioteca JavaScript amplamente usada para simplificar interações com elementos da página.

\begin{lstlisting}[language=JavaScript]
function elementHover(element, symbol) {
  $(element).on("mouseenter", function() {
    $(this).text(symbol + " " + $(this).text());
  });

  $(element).on("mouseleave", function() {
    if ($(this).text().startsWith(symbol)) {
      $(this).text($(this).text().substring(symbol.length));
    }
  });
}

elementHover(bio, "#");
elementHover(titleWeb, "#");
elementHover(titleCalculator, "##");
elementHover(titleDotnet, "#");
elementHover(titleFox, "##");
elementHover($("#title-personal"), "#");
elementHover($("#title-bash"), "##");
elementHover($("#bashCpu"), "###");
elementHover($("#bashDisk"), "###");
elementHover($("#bashProj"), "###");
\end{lstlisting}

Este código adiciona um efeito que prefixa os elementos de texto com símbolos específicos, como "\#" ou "\#\#", ao passar o cursor sobre eles, tornando-os mais destacados. Quando o cursor é removido, os símbolos são removidos. Isso é uma técnica usada para chamar a atenção dos visitantes para informações importantes na página.

\newpage

\subsubsection{Efeito de Destaque em Pontos de Bala}
Além disso, também adicionei um efeito de destaque a pontos de bala, que são atualizados automaticamente a cada poucos segundos.

\begin{lstlisting}[language=JavaScript]
// bullets
elementHover(bullet, "*");

// Change bullet point every couple of seconds:
const bulletPoints = ["Lover of programming.", "Linux Developer", "Web Developer"];
let indexBulletPoint = 0;

function updateBulletPoint() {
  if (indexBulletPoint < bulletPoints.length) {
    bullet.text(bulletPoints[indexBulletPoint]);
    indexBulletPoint++;
  } else {
    indexBulletPoint = 0;
    bullet.text(bulletPoints[indexBulletPoint]);
  }
}
const bulletPointInterval = setInterval(updateBulletPoint, 3000);
console.log(bulletPointInterval);
\end{lstlisting}

Aqui, os pontos de bala são destacados com um asterisco "*" quando o cursor está sobre eles. Além disso, a função `updateBulletPoint` é usada para alternar automaticamente entre mensagens de destaque em intervalos de três segundos.

\subsubsection{Carrossel de Imagens}
Implementei um carrossel de imagens em algumas seções do website, permitindo aos visitantes clicar para ver diferentes imagens e parágrafos. Por exemplo, no projeto "Calculator" e na seção "Foxy the Fox".

\begin{lstlisting}[language=JavaScript]
// Calculator:
// show image 2 on click
$('#imgCalc1').on("click", function() {
  $(this).addClass("hidden");
  $('#imgCalc2').removeClass("hidden");
});

// show image 1 on click
$('#imgCalc2').on("click", function() {
  $(this).addClass("hidden");
  $('#imgCalc1').removeClass("hidden");
});

// carousel paragraphs
// show paragraph 2 on click
$('#pCalc1').on('click', function() {
  $(this).addClass('hidden');
  $('#pCalc2').removeClass('hidden');
});

// show paragraph 1 on click
$('#pCalc2').on('click', function() {
  $(this).addClass('hidden');
  $('#pCalc1').removeClass('hidden');
}

// Foxy the Fox:
// ...
\end{lstlisting}

Nessas seções, os visitantes podem clicar nas imagens ou parágrafos para alternar entre diferentes visualizações, oferecendo uma experiência mais interativa e envolvente.

Essas funcionalidades interativas contribuem para a usabilidade e atração do website, tornando-o mais envolvente para os visitantes.

\subsection{Frameworks e Bibliotecas}
No projeto, dependi das seguintes tecnologias e recursos:

\begin{itemize}
    \item Framework jQuery: Utilizado para facilitar interações com elementos da página, como o carrossel de imagens.
    \item Font Awesome: Utilizado para incorporar ícones de redes sociais e outros elementos visuais.
    \item Tailwindcss: Utilizado como framework para criar o estilo gráfico da página ao invés de CSS tradicional
\end{itemize}

\subsection{Webgrafia de Recursos}
Recorri aos seguintes recursos externos no projeto:

\begin{itemize}
    \item Font Awesome: \url{fontawesome.com}
    \item jQuery: \url{jquery.com}
    \item tailwindcss \url{tailwindcss.com}
\end{itemize}

\section{Conclusão}
O meu website é uma representação eficaz das minhas habilidades e projetos. Utilizei HTML, CSS e JavaScript para criar uma experiência de usuário responsiva e interativa. A incorporação de ícones do Font Awesome e o menu de navegação fixo aprimoram a estética e a usabilidade do site. As funcionalidades interativas, como os efeitos de hover e o carrossel de imagens, proporcionam uma experiência envolvente para os visitantes.

\end{document}
